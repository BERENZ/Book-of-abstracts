\providecommand{\main}{..} 
\documentclass[\main/boa.tex]{subfiles}

\begin{document}

\section{Are we ready for Personalized Medicine?}

\begin{center}
  {\bf \index[a]{Łabaj Paweł P.} Paweł P. Łabaj$^{1^\star}$}
\end{center}

\vskip 0.3cm

\begin{affiliations}
\begin{enumerate}
\begin{minipage}{0.915\textwidth}
\centering
\item Department of Biotechnology, Boku University Vienna \\[-2pt]
\end{minipage}
\end{enumerate}
$^\star$Contact author: \href{mailto:pawel.labaj@boku.ac.at}{\nolinkurl{pawel.labaj@boku.ac.at}}\\
\end{affiliations}

\vskip 0.5cm

\begin{minipage}{0.915\textwidth}
\keywords medicine; exposome; biomedical data, Bioconductor
\end{minipage}

\vskip 0.8cm

In the era of fast-paced development of technology and services, there
are limitless opportunities for customization to meet specific user
needs. Over the next decade, as much as half of the proportion of health
care will shift from the hospital and clinic to the home and community.
With Personalized Medicine understood as prevention and treatment
strategies that take individual variability into account we need to
identify this individual variability via characterizing each person's
individual baseline health state instead of resorting to
population-based variable distributions. This health state baseline
cannot be, however, determined with use of just the classical medical
records. Recent technological advances have created opportunities to
harness additional sources of biomedical data on a real time basis, for
instance through the use of: (1) mobile medical devices for monitoring
dedicated health parameters (insulin, heart rate, etc), and (2)
wearables. The synergy of these two streams should provide a good
estimate of the health state baseline. In order to model estimated data
of health state baseline and future scenarios, it is imperative to
include an important, yet largely missing third component - the
exposome. This term cover all the exposures of an individual in a
lifetime. So far it was mostly connected with air quality, light,
climatic variations, ozone and volatile organic compounds. But we cannot
forget about the 'living' component of exposome. As dense human
environments such as cities account for over a half of the world
population (in EU 80\%) there is a need to build a molecular portrait of
cities in order to study what lives around us and how it affects our
health and wellbeing. There are, however, no dedicated tools, frameworks
or standards how to store, share, integrate these streams of data. There
is a lot to do as there are many open question and challenges which need
to be address in order to provide valuable input. Fortunately, there is
a very lively community around \textbf{R} Bioconductor project, which
provides tools for the analysis and comprehension of high-throughput
genomic and other biomedical data. This community should take a leading
role in the future development of solutions for Personalized Medicine.

\end{document}
