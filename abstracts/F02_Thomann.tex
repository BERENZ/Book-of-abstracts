\providecommand{\main}{..} 
\documentclass[\main/boa.tex]{subfiles}

\begin{document}

\section{SimonsSVM: A Fast and Scalable Support Vector Machine Implementation for
R}

\begin{center}
  {\bf Philipp Thomann$^{1^\star}$, Ingo Steinwart$^{1}$}
\end{center}

\vskip 0.3cm

\begin{affiliations}
\begin{enumerate}
\begin{minipage}{0.915\textwidth}
\centering
\item ISA / University of Stuttgart \\[-2pt]
\end{minipage}
\end{enumerate}
$^\star$Contact author: \href{mailto:philipp.thomann@mathematik.uni-stuttgart.de}{\nolinkurl{philipp.thomann@mathematik.uni-stuttgart.de}}\\
\end{affiliations}

\vskip 0.5cm

\begin{minipage}{0.915\textwidth}
\keywords support vector machine; machine learning; non-parametric classification;
non-parametric regression
\packages SimonsSVM
\end{minipage}

\vskip 0.8cm

Support vector machines (SVMs) are non-parametric methods for various
supervised learning scenarios like classification and regression. They
have been studied extensively both theoretically and practically in the
last 20 years. There are many well-known implementations also for
\textbf{R}, for instance the package \emph{e1071} provides bindings to
\emph{LIBSVM}. We present our recent package \emph{Simons' SVM} for
\textbf{R} with the following key features:

\begin{enumerate}
\def\labelenumi{\arabic{enumi}.}
\tightlist
\item
  Unprecented speed:
\end{enumerate}

\begin{itemize}
\tightlist
\item
  Compared to, e.g. \emph{LIBSVM}, training that includes 5-fold cross
  validation on a 10x10 hyper-parameter grid is about 300 times faster
  on, e.g.~a 1000 samples containing subset of the binary version of the
  classical covertype dataset.
\item
  Partitioning strategies further decrease training (and testing) time
  without sacrifying generalization. For example, the full covertype
  data set (about 523.000 samples) takes less than 9 min.
\item
  Partitioning even allows to attack huge problems. For instance the
  higgs data set (10 million samples) could be trained and tested in
  five hours.
\end{itemize}

\begin{enumerate}
\def\labelenumi{\arabic{enumi}.}
\setcounter{enumi}{1}
\tightlist
\item
  Inclusion of some standard learning scenarios:
\end{enumerate}

\begin{itemize}
\tightlist
\item
  (weighted) binary classification
\item
  multiclass classification (both AvA and OvA)
\item
  Neyman-Pearson-type classification
\item
  Least squares / quantile / expectile regression
\end{itemize}

\begin{enumerate}
\def\labelenumi{\arabic{enumi}.}
\setcounter{enumi}{2}
\tightlist
\item
  Flexible user interface:
\end{enumerate}

\begin{itemize}
\tightlist
\item
  Fully integrated cross validation let's the user focus on parameters
  he/she can understand.
\item
  The underlying implementation have more than twenty independent otions
  leading to an enormeous flexibility for the power user.
\item
  Meaningful default values, which in many cases makes a fine-adjustment
  unneccesary.
\item
  Comprehensive documentation.
\end{itemize}

Quick demo:

\begin{verbatim}
install.packages("SimonsSVM",
repos="http://www.isa.uni-stuttgart.de/software/R")
library(SimonsSVM)
d <- ttsplit(iris)
model <- svm(Species ~ ., d$train)
test(model, d$test)
\end{verbatim}

More information:

\url{http://www.isa.uni-stuttgart.de/software/R/demo.html}

\url{http://www.isa.uni-stuttgart.de/software/R/documentation.html}

\end{document}
