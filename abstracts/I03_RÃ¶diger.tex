\documentclass[11pt, a4paper]{article}
\usepackage[utf8]{inputenc}
\usepackage[T1]{fontenc}
\usepackage{eurosym}
\usepackage{amsfonts, amsmath, hanging, hyperref, parskip, times}
\usepackage[numbers]{natbib}
\usepackage[pdftex]{graphicx}
\hypersetup{
  colorlinks,
  linkcolor=black,
  urlcolor=black,
  citecolor=black
}

\let\section=\subsubsection
\newcommand{\pkg}[1]{{\normalfont\fontseries{b}\selectfont #1}}
\let\proglang=\textit
\let\code=\texttt
\renewcommand{\title}[1]{\begin{center}{\bf \LARGE #1}\end{center}}
\newcommand{\affiliations}{\footnotesize\centering}
\newcommand{\keywords}{\paragraph{Keywords:}}
\newcommand{\packages}{\paragraph{R packages:}}

\providecommand{\tightlist}{%
  \setlength{\itemsep}{0pt}\setlength{\parskip}{0pt}}

\setlength{\topmargin}{-15mm}
\setlength{\oddsidemargin}{-2mm}
\setlength{\textwidth}{165mm}
\setlength{\textheight}{250mm}


\begin{document}
\pagestyle{empty}

\title{R as an Environment for the Reproducible Analysis of DNA Amplification
Experiments}

\begin{center}
  {\bf Stefan Rödiger$^{1^\star}$, Michal Burdukiewic$^{2}$, Peter Schierac$^{1}$}
\end{center}

\vskip 0.3cm

\begin{affiliations}
\begin{enumerate}
\begin{minipage}{0.915\textwidth}
\centering
\item Institute of Biotechnology, Brandenburg University of Technology
Cottbus-Senftenberg \\[-2pt]
\item Department of Genomics, University of Wroclaw \\[-2pt]
\end{minipage}
\end{enumerate}
$^\star$Contact author: \href{mailto:stefan.roediger@b-tu.de}{\nolinkurl{stefan.roediger@b-tu.de}}\\
\end{affiliations}

\vskip 0.5cm

\begin{minipage}{0.915\textwidth}
\keywords bioinfrmatics; qPCR; digital PCR; reproducible research;non-linear
regression; smoothing; pre-processingI;
\packages chipPCR; dpcR; MBmca; RDML; rkwarddev; qpcR
\end{minipage}

\vskip 0.8cm

Quantitative PCR (qPCR), digital PCR (dPCR), quantitative isothermal
amplification (qIA) and melting curve analysis (MCA) are key
technologies in molecular diagnostics, forensics and life sciences.
However, software for the biostatistical data analysis of such
technologies is either tied to a specific task or part of a monolithic
closed source software. This limits the reproducibility of computations
and gives no control over the analysis algorithms.

We contributed bioinformatics software tools for reproducible and
transparent data analysis. Analysis pipelines consisting of statistical
procedures, raw data preprocessing, analysis, charts and report
generation are implemented for increasingly demanded reproducible
research. This is achieved with our packages \emph{MBmca},
\emph{chipPCR}, \emph{dpcR} and \emph{RDML}. We implemented selected
findings for the \textbf{R} GUI/IDE RKWard and as \emph{shiny} web
browser applications. For rapid prototyping of RKWard plugins we used
the \emph{rkwarddev} package.

In our exemplary workflows we show analyzers for qPCR, qIA, MCA and dPCR
experiments, which can be build in short development cycles. Our
software is targeted at users who develop novel devices or users who
wish to analyze raw and unprocessed data.

\end{document}
