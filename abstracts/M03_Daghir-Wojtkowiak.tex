\providecommand{\main}{..} 
\documentclass[\main/boa.tex]{subfiles}

\begin{document}

\section{Pharmacokinetics-driven modeling of metabolomics data}

\begin{center}
  {\bf Emilia Daghir-Wojtkowiak$^{1^\star}$, Paweł Wiczling$^{1}$, Małgorzata Waszczuk-Jankowska$^{1}$, Roman Kaliszan$^{1}$, Michał J. Markuszewski$^{1}$}
\end{center}

\vskip 0.3cm

\begin{affiliations}
\begin{enumerate}
\begin{minipage}{0.915\textwidth}
\centering
\item Department of Biopharmaceutics and Pharmacodynamics, Medical University
of Gdańsk \\[-2pt]
\end{minipage}
\end{enumerate}
$^\star$Contact author: \href{mailto:emilia.daghir@gmail.com}{\nolinkurl{emilia.daghir@gmail.com}}\\
\end{affiliations}

\vskip 0.5cm

\begin{minipage}{0.915\textwidth}
\keywords pharmacokinetics; metabolomics; Bayesian analysis; ROC; cancer
\packages runjags; coda; caTools; MCMCglmm; caret; Hmisc; lattice
\end{minipage}

\vskip 0.8cm

Metabolomics is a dynamically developing research area utilizing
high-throughput analytical techniques to detect spectrum of changes in
metabolites' levels. However, technological improvements designed for
analytical systems are not parallel with the development/application of
computational methods which would allow for more efficient elucidation
of knowledge from large datasets. In this study, we introduced the
concept of pharmacokinetics-driven modelling of targeted metabolomics
data comprising nucleoside and creatinine concentration measurements in
urine of healthy and cancer patients. An approach using Bayesian
analysis was used for the estimation of model parameters. The
classification performance of the proposed model was summarized via area
under the ROC (Receiver-Operator Curve), sensitivity and specificity
using external validation. Cancer was associated with an increase in
methylthioadenosine/creatinine excretion rate by a factor of 1.82
(1.33--2.47). Age influenced nucleosides/creatinine excretion rates for
all nucleosides in the same direction with likely sex-related
differences among several nucleosides' concentrations. The individual a
posteriori prediction of patient classification as area under the ROC
was 0.58 (0.5-0.68) with sensitivity and specificity of 0.63
(0.46--0.76) and 0.58 (0.45--0.68), respectively suggesting limited
usefulness of 13 nucleosides/creatinine measurements in predicting the
disease in this population. Pharmacokinetic-based approach in
metabolomics may be useful in understanding the data when searching for
potential disease indicators.

\end{document}
