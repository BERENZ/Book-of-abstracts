\documentclass[11pt, a4paper]{article}
\usepackage[utf8]{inputenc}
\usepackage[T1]{fontenc}
\usepackage{eurosym}
\usepackage{amsfonts, amsmath, hanging, hyperref, parskip, times}
\usepackage[numbers]{natbib}
\usepackage[pdftex]{graphicx}
\hypersetup{
  colorlinks,
  linkcolor=black,
  urlcolor=black,
  citecolor=black
}

\let\section=\subsubsection
\newcommand{\pkg}[1]{{\normalfont\fontseries{b}\selectfont #1}}
\let\proglang=\textit
\let\code=\texttt
\renewcommand{\title}[1]{\begin{center}{\bf \LARGE #1}\end{center}}
\newcommand{\affiliations}{\footnotesize\centering}
\newcommand{\keywords}{\paragraph{Keywords:}}
\newcommand{\packages}{\paragraph{R packages:}}

\providecommand{\tightlist}{%
  \setlength{\itemsep}{0pt}\setlength{\parskip}{0pt}}

\setlength{\topmargin}{-15mm}
\setlength{\oddsidemargin}{-2mm}
\setlength{\textwidth}{165mm}
\setlength{\textheight}{250mm}


\begin{document}
\pagestyle{empty}

\title{brms: An R Package for Bayesian Multilevel Models using Stan}

\begin{center}
  {\bf Paul-Christian Buerkner$^{1^\star}$}
\end{center}

\vskip 0.3cm

\begin{affiliations}
\begin{enumerate}
\begin{minipage}{0.915\textwidth}
\centering
\item The University of Münster \\[-2pt]
\end{minipage}
\end{enumerate}
$^\star$Contact author: \href{mailto:paul.buerkner@gmail.com}{\nolinkurl{paul.buerkner@gmail.com}}\\
\end{affiliations}

\vskip 0.5cm

\begin{minipage}{0.915\textwidth}
\keywords Bayesian inference; multilevel model; MCMC, Stan
\packages brms
\end{minipage}

\vskip 0.8cm

The \emph{brm} package implements Bayesian multilevel models in
\textbf{R} using the probabilistic programming language Stan. A wide
range of distributions and link functions are supported, allowing to fit
-- among others -- linear, robust linear, binomial, Poisson, survival,
ordinal, zero-inflated, hurdle, and even non-linear models all in a
multilevel context. Further modeling options include autocorrelation of
the response variable, user defined covariance structures, censored
data, as well as meta-analytic standard errors. Prior specifications are
flexible and explicitly encourage users to apply prior distributions
that actually reflect their beliefs. In addition, model fit can easily
be assessed and compared with the Watanabe-Akaike Information Criterion
and leave-one-out cross-validation.

\end{document}
