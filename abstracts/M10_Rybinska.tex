\providecommand{\main}{..} 
\documentclass[\main/boa.tex]{subfiles}

\begin{document}

\section{R as an effective data mining tool in chemistry}

\begin{center}
  {\bf Anna Rybińska$^{1^\star}$, Katarzyna Odziomek$^{1}$, Tomasz Puzyn$^{1}$}
\end{center}

\vskip 0.3cm

\begin{affiliations}
\begin{enumerate}
\begin{minipage}{0.915\textwidth}
\centering
\item Laboratory of Environmental Chemometrics, University of Gdansk \\[-2pt]
\end{minipage}
\end{enumerate}
$^\star$Contact author: \href{mailto:rybinska@qsar.eu.org}{\nolinkurl{rybinska@qsar.eu.org}}\\
\end{affiliations}

\vskip 0.5cm

\begin{minipage}{0.915\textwidth}
\keywords data mining; predictive modeling; ionic liquids; chemometrics
\packages ggplot2; stats; dendextend; matrixStats; Matrix; MASS; klaR; cluster;
dplyr; FWDselect
\end{minipage}

\vskip 0.8cm

The vast amount of digital information generated every day in social
media, industry and academia necessitates the use of advanced techniques
enabling the processing, analysis and interpretation of data. A
significant amount of data is generated during various types of chemical
experiments. In this work, we present a workflow for predictive modeling
of physical-chemical properties of a diverse group of chemicals (ionic
liquids). Using selected chemometric methods, available in the most
popular \textbf{R} packages, we analyze the chemical data, model key
physical-chemical properties and validate the results. Moreover, the
presented approach enables the evaluation of the raw data quality. We
choose two unsupervised methods for data exploration: hierarchical
cluster analysis (HCA) and principal component analysis (PCA). Multiple
linear regression technique (MLR) was chosen as the example of a
modeling technique.

\end{document}
