\providecommand{\main}{..} 
\documentclass[\main/boa.tex]{subfiles}

\begin{document}
\pagestyle{empty}

\section{Modules in R}

\begin{center}
  {\bf Sebastian Warnholz$^{1^\star}$}
\end{center}

\vskip 0.3cm

\begin{affiliations}
\begin{enumerate}
\begin{minipage}{0.915\textwidth}
\centering
\item INWT-Statistics \\[-2pt]
\end{minipage}
\end{enumerate}
$^\star$Contact author: \href{mailto:wahani@gmail.com}{\nolinkurl{wahani@gmail.com}}\\
\end{affiliations}

\vskip 0.5cm

\begin{minipage}{0.915\textwidth}
\keywords programming; functional-programming
\packages modules; import; parallel
\end{minipage}

\vskip 0.8cm

In this talk I present the idea of modules inside the \textbf{R}
language. Modules are an organizational unit for source code. The key
idea of this package is to provide a unit which is self contained,
i.e.~has it's own scope. The main and most reliable infrastructure for
such organizational units of source code in the \textbf{R} ecosystem is
a package. Compared to a package modules can be considered ad hoc, but
-- in the sense of an \textbf{R} package -- self contained. Furthermore
modules typically consist of one file; in contrast to a package which
can wrap an arbitrary number of files. Inside of packages modules act
more like objects, as in object-oriented-programming. In this talk I
cover basic use cases in parallel computing and good coding practices.

\end{document}
