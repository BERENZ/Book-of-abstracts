\providecommand{\main}{..} 
\documentclass[\main/boa.tex]{subfiles}

\begin{document}

\section{Modelling the distrubution of the bryophytes in different spatial scales}

\begin{center}
  {\bf Sylwia Wierzcholska$^{1^\star}$, Marcin K. Dyderski$^{2, 3}$, Andrzej M. Jagodziński$^{2, 3}$}
\end{center}

\vskip 0.3cm

\begin{affiliations}
\begin{enumerate}
\begin{minipage}{0.915\textwidth}
\centering
\item Białowieża Geobotanical Station, University of Warsaw \\[-2pt]
\item Institute of Dendrology, Polish Academy of Sciences \\[-2pt]
\item Department of Game Management and Forest Protection, Poznań University
of Life Sciences \\[-2pt]
\end{minipage}
\end{enumerate}
$^\star$Contact author: \href{mailto:sylwia.wierzcholska@gmail.com}{\nolinkurl{sylwia.wierzcholska@gmail.com}}\\
\end{affiliations}

\vskip 0.5cm

\begin{minipage}{0.915\textwidth}
\keywords species distribution model; biodiversity; plant ecology; predictive
modelling; mosses; forest
\packages dismo; rasterVis
\end{minipage}

\vskip 0.8cm

Species distribution models (SDM) are mathematical models describing
distribution of species within environmental (ecological) and
geographical space. These models are used for biodiversity conservation
in two ways: to find out threatened species requirements and to predict
spread of invasive species. We aimed to model distribution of model
bryophyte species -- \emph{Dicranum viride} -- in Poland and in Europe
and compare its ecological niches obtained by these two models. We chose
\emph{D. viride}, as this easily-recognizable species is a subject of
Natura 2000 protection and an ancient forest species. Thus, it may be an
umbrella species for numerous bryophyte species occurring on decaying
wood and in old woodlands. We used data from Global Biodiversity
Information Facility (\url{http://www.gbif.org}), published papers and
herbarium collections to find out complete information about \emph{D.
viride} localities. As most of data about species distribution for large
areas is presence-only data, we used MaxEnt model from \emph{dismo}
package, which is developed to processing this type of input data. As
the explanatory variables we used 19 bioclimatic statistics from
WordlClim database, available in 2.5' grid and, in case of model for
Poland only, data about share of old (\textgreater{}100 years old)
forests within grid square. We also analyzed data about phorophyte
species and collection year. Model developed by MaxEnt is a probalistic
model, which due to Receiver-Operator Curve allow to manage the
threshold of species occurence, and thus -- its restrictiveness. Our
model has shown importance of particular bioclimatic variables and
potential distribution of species. Obtained results allow us to conclude
about climatic requirements of species studied and its potential
habitats, where species may be found or may be protected ex situ. SDMs
are very usefull tool for plant geography, biodiversity conservation and
ecology.

\end{document}
