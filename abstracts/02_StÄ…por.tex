\documentclass[11pt, a4paper]{article}
\usepackage[utf8]{inputenc}
\usepackage[T1]{fontenc}
\usepackage{eurosym}
\usepackage{amsfonts, amsmath, hanging, hyperref, parskip, times}
\usepackage[numbers]{natbib}
\usepackage[pdftex]{graphicx}
\hypersetup{
  colorlinks,
  linkcolor=black,
  urlcolor=black,
  citecolor=black
}

\let\section=\subsubsection
\newcommand{\pkg}[1]{{\normalfont\fontseries{b}\selectfont #1}}
\let\proglang=\textit
\let\code=\texttt
\renewcommand{\title}[1]{\begin{center}{\bf \LARGE #1}\end{center}}
\newcommand{\affiliations}{\footnotesize\centering}
\newcommand{\keywords}{\paragraph{Keywords:}}
\newcommand{\packages}{\paragraph{R packages:}}

\providecommand{\tightlist}{%
  \setlength{\itemsep}{0pt}\setlength{\parskip}{0pt}}

\setlength{\topmargin}{-15mm}
\setlength{\oddsidemargin}{-2mm}
\setlength{\textwidth}{165mm}
\setlength{\textheight}{250mm}


\begin{document}
\pagestyle{empty}

\title{Heteroscedastic Discriminant Analysis and its integration into `mlR'
package for uniform machine learning}

\begin{center}
  {\bf Katarzyna Stąpor$^{1^\star}$}
\end{center}

\vskip 0.3cm

\begin{affiliations}
\begin{enumerate}
\begin{minipage}{0.915\textwidth}
\centering
\item Institute of Computer Science, Silesian Technical University \\[-2pt]
\end{minipage}
\end{enumerate}
$^\star$Contact author: \href{mailto:katarzyna.stapor@polsl.pl}{\nolinkurl{katarzyna.stapor@polsl.pl}}\\
\end{affiliations}

\vskip 0.5cm

\begin{minipage}{0.915\textwidth}
\keywords discriminant analysis; machine learning; heteroscedasticity
\packages mlR
\end{minipage}

\vskip 0.8cm

The \emph{mlR} package (machine learning in \textbf{R}) offers a unified
interface to access various machine learning algorithms from other
packages in \textbf{R}. This framework provides supervised methods like
classification, regression and survival analysis along with their
corresponding evaluation and optimization methods, as well as
unsupervised methods like clustering. It is written in a way that you
can extend it yourself or deviate from the implemented convenience
methods and construct your own complex experiments or algorithms. As an
example, it will be shown how to integrate into the \emph{mlR} package
the new, proposed by us learner, the Heteroscedastic Discriminant
Analysis (HDA), being the extension of the classical Fisher Linear
Discriminant Analysis (FDA), implemented in the \emph{base} package. HDA
is the extension of FDA for dealing with the case of unequal covariance
matrices in the populations, the situation that occurs very often in
practice. The new implemented permutation test for testing the equality
of covariance matrices will be also presented. Integration the new
learner into \emph{mlR} requires defining the learner itself with the
name, description, parameters, and a few other things, then providing
the function that calls the learner function and builds the model given
the data, and finally, a prediction function that returns predicted
values given new data. The example of usage the new HDA learner on the
real world credit dataset will also be presented.

\end{document}
