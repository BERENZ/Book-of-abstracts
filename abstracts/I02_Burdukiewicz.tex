\providecommand{\main}{..} 
\documentclass[\main/boa.tex]{subfiles}

\begin{document}

\section{N-gram analysis of biological sequences in R}

\begin{center}
  {\bf \index[a]{Michał Burdukiewicz}$^{1^\star}$, \index[a]{Piotr Sobczyk}$^{2}$, \index[a]{Małgorzata Kotulska}$^{3}$, \index[a]{Paweł Mackiewicz}$^{1}$}
\end{center}

\vskip 0.3cm

\begin{affiliations}
\begin{enumerate}
\begin{minipage}{0.915\textwidth}
\centering
\item Department of Genomics, University of Wrocław \\[-2pt]
\item Institute of Mathematics, Wrocław University of Technology \\[-2pt]
\item Department of Biomedical Engineering, Wrocław University of Technology \\[-2pt]
\end{minipage}
\end{enumerate}
$^\star$Contact author: \href{mailto:michalburdukiewicz@gmail.com}{\nolinkurl{michalburdukiewicz@gmail.com}}\\
\end{affiliations}

\vskip 0.5cm

\begin{minipage}{0.915\textwidth}
\keywords n-gram; k-mer; feature selection; proteomics
\packages biogram; ranger; slam
\end{minipage}

\vskip 0.8cm

N-grams (k-mers) are vectors of \(n\) characters derived from input
sequences, widely used in genomics, transcriptomics and proteomics.
Despite the continuous interest in the sequence analysis, there are only
a few tools tailored for comparative n-gram studies. Furthermore, the
volume of n-gram data is usually very large, making its analysis in
\textbf{R} especially challenging.

The CRAN package \emph{biogram} facilitates incorporating n-gram data in
the \textbf{R} workflows. Aside from the efficient extraction and
storage of n-grams, the package offers also a feature selection method
designed specifically for this type of data. QuiPT (Quick Permutation
Test) uses several filtering criteria such as information gain (mutual
information) to choose significant n-grams. To speed up the computation
and allow precise estimation of small \(p\)-values, QuiPT uses
analytically derived distributions instead of a large number of
permutations. In addition to this, \emph{biogram} contains tools
designed for reducing the dimensionality of the amino acid alphabet,
further scaling down the feature space.

To illustrate the usage of n-gram data in the analysis of biological
sequences, we present two case studies performed solely in \textbf{R}.
The first, prediction of amyloids, short proteins associated with the
number of clinical disorders as Alzheimer's or Creutzfeldt-Jakob's
diseases, employs random forests trained on n-grams. The second,
detection of signal peptides orchestrating an extracellular transport of
proteins, utilizes more complicated probabilistic framework (hidden
semi-Markov model) but still uses n-gram data for training.

\end{document}
