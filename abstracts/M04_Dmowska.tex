\providecommand{\main}{..} 
\documentclass[\main/boa.tex]{subfiles}

\begin{document}

\section{R as a tool for geospatial modeling in large dataset - example of
dasymetric modeling at a continental scale (United States)}

\begin{center}
  {\bf Anna Dmowska$^{1^\star}$}
\end{center}

\vskip 0.3cm

\begin{affiliations}
\begin{enumerate}
\begin{minipage}{0.915\textwidth}
\centering
\item Adam Mickiewicz University \\[-2pt]
\end{minipage}
\end{enumerate}
$^\star$Contact author: \href{mailto:dmowska@amu.edu.pl}{\nolinkurl{dmowska@amu.edu.pl}}\\
\end{affiliations}

\vskip 0.5cm

\begin{minipage}{0.915\textwidth}
\keywords geospatial data; large dataset; dasymetric modeling
\packages rgrass7; rgdal; RSQLite; sp
\end{minipage}

\vskip 0.8cm

Geospatial raster datasets are usually large data structures. Thus its
processing requires algorithms that are both efficient and fully
automatic. The philosophy of development of GIS software offers many
solutions to a user. These solutions are conceptually limited which
means that they perform well in their native tasks but are difficult to
extend if the problem goes beyond its foundations. In such situations
computationally complex tasks require tailored software which is
difficult to develop without programming experiences. \textbf{R}
language offers extensive amount of tools designed to work with
geospatial data like \emph{sp} library, as well as bindings to external
data sources like \emph{rgrass7}, \emph{rgdal}, \emph{RSQLite}. Flexible
nature of geospatial objects in \textbf{R} allows to create a fully
automatic procedure that complements GIS software without long-term
programing experiences.

Here we present an automatic procedure which was designed to work on
high resolution datasets at a continental scale: 11 million of records
in tabular data and over 8 billion of grid cells. To process such amount
of data we automated our procedure by dividing it into few steps:

\begin{enumerate}
\def\labelenumi{\arabic{enumi}.}
\tightlist
\item
  import and store Census data in SQLite database,
\item
  preprocessing of geospatial data performed in GRASS GIS software,
\item
  performing dasymetric modeling in \textbf{R},
\item
  propagation model for geospatial data and export geospatial data from
  \textbf{R} to GIS format to map population distribution.
\end{enumerate}

Our results show that:

\begin{enumerate}
\def\labelenumi{\arabic{enumi}.}
\tightlist
\item
  \textbf{R} can perform such a calculation in a reasonable time (60
  hours for the entire U.S.),
\item
  no additional intermediate layers or steps are required,
\item
  procedure is flexible: it can be reproduced regardless of the data
  scale.
\end{enumerate}

\end{document}
