\providecommand{\main}{..} 
\documentclass[\main/boa.tex]{subfiles}

\begin{document}
\pagestyle{empty}

\section{Search phrases in e-commerce platform allegro.pl - big data analysis
using SparkR}

\begin{center}
  {\bf Liliana Pięta$^{1, 2}$, Mariusz Strzelecki$^{1}$, Anna Wróblewska$^{1, 3^\star}$}
\end{center}

\vskip 0.3cm

\begin{affiliations}
\begin{enumerate}
\begin{minipage}{0.915\textwidth}
\centering
\item Allegro \\[-2pt]
\item Cracow University of Economics \\[-2pt]
\item Warsaw University of Technology \\[-2pt]
\end{minipage}
\end{enumerate}
$^\star$Contact author: \href{mailto:anna.wroblewska@allegrogroup.com}{\nolinkurl{anna.wroblewska@allegrogroup.com}}\\
\end{affiliations}

\vskip 0.5cm

\begin{minipage}{0.915\textwidth}
\keywords text mining; spark; big data
\packages sparkR; wordcloud; tm; rms
\end{minipage}

\vskip 0.8cm

\textbf{R} is a powerful tool for data visualization and analysis but
it's speed is limited to only one CPU speed. It stays in the opposite
with current Big Data trend that focuses on gathering and analysing
huge, terabyte-scale datasets in clustered environment. Combining
\textbf{R} and Spark technology we can synergize power of the both tools
and we can efficiently work on huge datasets. During the presentation we
will show you our approach to analyze search phrases in the biggest
polish e-commerce marketplace platform Allegro.pl. We try text mining
methods on phrases that our users type into the search box. Moreover we
analyse transactions and user search events and try to find out the
associations between searching phrase and successful transactional
event. We will focus on challenges we solved during adapting
\emph{sparkR} in Allegro and share difficulties we experienced.

\end{document}
