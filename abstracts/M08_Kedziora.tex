\providecommand{\main}{..} 
\documentclass[\main/boa.tex]{subfiles}

\begin{document}

\section{Wrestling with big data in forestry: use of R in Scots pine site index
analysis.}

\begin{center}
  {\bf Wojciech Kędziora$^{1^\star}$, Robert Tomusiak$^{2}$}
\end{center}

\vskip 0.3cm

\begin{affiliations}
\begin{enumerate}
\begin{minipage}{0.915\textwidth}
\centering
\item Department of Forest Management Planning, Geomatics and Forest
Economics, Warsaw University of Life Sciences -- SGGW \\[-2pt]
\item Laboratory of Dendrometry and Forest Productivity, Warsaw University of
Life Sciences -- SGGW \\[-2pt]
\end{minipage}
\end{enumerate}
$^\star$Contact author: \href{mailto:wojciech.kedziora@wl.sggw.pl}{\nolinkurl{wojciech.kedziora@wl.sggw.pl}}\\
\end{affiliations}

\vskip 0.5cm

\begin{minipage}{0.915\textwidth}
\keywords national forest inventory; permanent plots; site index
\end{minipage}

\vskip 0.8cm

National Forest Inventory (NFI) supplies high resolution data on forests
of all property forms located throughout the country. These data are
collected from the circular sample plots uniformly distributed in grid
of squares 4 by 4 km. In Poland, around \textasciitilde{}27,900 sample
plots were established. Recently, the second measuring cycle NFI was
completed, which, in addition to a potentially wide range of spatial
analysis, creates a unique opportunity to analyze the change in the
pattern of tree growth through time, giving opportunity to describe the
reaction of species to changes of environmental conditions. NFI data
give a great opportunity to determine one of the most important forest
habitat's productivity factors -- site index. It is often expressed as
the average height of the trees of the target species at a given age.
Determination of the site index for the dominant species in the stand
allows one to characterize its growth potential in examined habitat
conditions. The project envisages the use of empirical data collected in
NFI to determine the structure of pine stands site index in Poland.
Relatively high number of sample plots and complicated computations as
well as future need for geostatistical analyses were reasons to choose
\textbf{R} environment to work successfully with forestry big data.

\end{document}
