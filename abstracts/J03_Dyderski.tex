\providecommand{\main}{..} 
\documentclass[\main/boa.tex]{subfiles}

\begin{document}

\section{Is forest a pharmacy? - problems with data analyses}

\begin{center}
  {\bf Marcin K. Dyderski$^{1, 2^\star}$}
\end{center}

\vskip 0.3cm

\begin{affiliations}
\begin{enumerate}
\begin{minipage}{0.915\textwidth}
\centering
\item Institute of Dendrology, Polish Academy of Sciences \\[-2pt]
\item Department of Game Management and Forest Protection, Poznan University
of Life Sciences \\[-2pt]
\end{minipage}
\end{enumerate}
$^\star$Contact author: \href{mailto:Marcin.Dyderski@gmail.com}{\nolinkurl{Marcin.Dyderski@gmail.com}}\\
\end{affiliations}

\vskip 0.5cm

\begin{minipage}{0.915\textwidth}
\keywords forest ecology; data mining; visualisation; linear models
\packages caret; randomforest; ggplot2; hglm; sp; rgeos
\end{minipage}

\vskip 0.8cm

\textbf{Scientific supervisor}: Dr.~Andrzej M. Jagodzinski

Forest ecology is a specific branch of ecology, where due to extent
spatial scale and large dimensions of individual objects we often need a
special approaches. The aim of this presentation is to show a few case
studies representing frequent analytical problems in forest ecology and
its sources. I will present problem with unknown variability of studied
phenomena on the example of individual biomass of herbaceous forest
understory plant species. Although minimal sample size was unknown, the
biggest analytical problem was discontinuity of species occurrences,
connected with their specific biology. Thus, instead of two-factor
analyses we performed simple ANOVA and exploratory data analysis to find
specific patterns. In case of the study on alien species natural
regeneration, due to high amount of zero values and lack of good
explanatory variables, we were forced to simplify our densities data
into presence-absence and use logistic model to find interactions
between type of tree stand and distance from the seed source. However,
this interaction is hard to find without graphical data exploration. I
also will present the study in coarser spatial scale, based mainly on
published data on ancient woodland indicator species. This study is a
simple analyses of species richness and use of random forest model to
analyze highly-collinear variables -- land-use types within grid square.
Here I will show that simple (ANOVA) and more sophisticated (Poisson
Hierarchic GLM with SAR) statistical approach lead into the same
conclusions. I will also discuss potential sources of outliers and
artifacts, which may be important threat for results interpretation.

\end{document}
