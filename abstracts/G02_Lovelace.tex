\providecommand{\main}{..} 
\documentclass[\main/boa.tex]{subfiles}

\begin{document}

\section{stplanr: an R package for transport planning}

\begin{center}
  {\bf Robin Lovelace$^{1^\star}$}
\end{center}

\vskip 0.3cm

\begin{affiliations}
\begin{enumerate}
\begin{minipage}{0.915\textwidth}
\centering
\item University of Leeds \\[-2pt]
\end{minipage}
\end{enumerate}
$^\star$Contact author: \href{mailto:r.lovelace@leeds.ac.uk}{\nolinkurl{r.lovelace@leeds.ac.uk}}\\
\end{affiliations}

\vskip 0.5cm

\begin{minipage}{0.915\textwidth}
\keywords transport; GIS; modelling; visualisation
\packages stplanr; leaflet; shiny
\end{minipage}

\vskip 0.8cm

\emph{stplanr} was developed to solve a real world problem: how to
convert official data on travel patterns into geographic objects that
could be plotted on a map and analysed using GIS? Over time the package
has evolved to include a number of other functions. Analysis of road
traffic casualty data, various routing algorithms, `travel watershed'
analysis and access to Google's Travel Matrix are all possible. This
paper traces the development of these capabilities with a focus on
applied case studies and reproducible examples.

\end{document}
