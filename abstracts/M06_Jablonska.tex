\providecommand{\main}{..} 
\documentclass[\main/boa.tex]{subfiles}

\begin{document}

\section{cgmisc: enhanced genome-wide association analyses and visualization}

\begin{center}
  {\bf \index[a]{Jagoda Jabłońska}$^{1^\star}$, \index[a]{Marcin Kierczak}$^{2}$, \index[a]{Simon Forsberg}$^{3}$}
\end{center}

\vskip 0.3cm

\begin{affiliations}
\begin{enumerate}
\begin{minipage}{0.915\textwidth}
\centering
\item University of Warsaw \\[-2pt]
\item Uppsala University \\[-2pt]
\item Swedish University of Agricultural Sciences \\[-2pt]
\end{minipage}
\end{enumerate}
$^\star$Contact author: \href{mailto:jagoda100jablonska@gmail.com}{\nolinkurl{jagoda100jablonska@gmail.com}}\\
\end{affiliations}

\vskip 0.5cm

\begin{minipage}{0.915\textwidth}
\keywords GWAS; bioinformatics; visualization; sequencing
\packages cgmisc; GenABEL
\end{minipage}

\vskip 0.8cm

High-throughput genotyping and sequencing technologies facilitate
studies of complex genetic traits and provide new research
opportunities. The increasing popularity of genome-wide association
studies (GWAS) leads to the discovery of new associated loci and a
better understanding of the genetic architecture underlying not only
diseases, but also other monogenic and complex phenotypes. Several
softwares are available for performing GWAS analyses, \textbf{R}
environment being one of them. We present \emph{cgmisc}, \textbf{R}
package that enables enhanced data analysis and visualization of results
from GWAS. The package contains several utilities and modules that
complement and enhance the functionality of the existing software. It
also provides several tools for advanced visualization of genomic data
and utilizes the power of the \textbf{R} language to aid in preparation
of publication-quality figures. Some of the package functions are
specific for the domestic dog (\emph{Canis familiaris}) data but are
easy to adjust for analysing other species.

\end{document}
