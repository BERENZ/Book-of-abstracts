\providecommand{\main}{..} 
\documentclass[\main/boa.tex]{subfiles}

\begin{document}

\section{Simulation of complex synthetic data with the R package simPop}

\begin{center}
  {\bf \index[a]{Templ Matthias} Matthias Templ$^{1^\star}$}
\end{center}

\vskip 0.3cm

\begin{affiliations}
\begin{enumerate}
\begin{minipage}{0.915\textwidth}
\centering
\item Vienna University of Technology \\[-2pt]
\end{minipage}
\end{enumerate}
$^\star$Contact author: \href{mailto:matthias.templ@gmail.com}{\nolinkurl{matthias.templ@gmail.com}}\\
\end{affiliations}

\vskip 0.5cm

\begin{minipage}{0.915\textwidth}
\keywords synthetic data; complex sample designs; population data; prediction
\packages \index[p]{popSim} popSim
\end{minipage}

\vskip 0.8cm

The production of synthetic datasets has been proposed as a statistical
disclosure control solution to generate public use files out of
protected data. This is also a tool to create ``augmented datasets'' to
serve as input for micro-simulation models, and -- more generally -- the
synthetic data sets can be used for design-based simulation studies in
general. The performance and acceptability of such a tool relies heavily
on the quality of the synthetic data, i.e.~on the statistical similarity
between the synthetic and the true population of interest. Multiple
approaches and tools have been developed to generate synthetic data.
These approaches can be categorized into three main groups: synthetic
reconstruction, combinatorial optimization, and model-based generation.
We introduce \emph{simPop}, an open source data synthesizer.
\emph{simPop} is a user-friendly \textbf{R} package based on a modular
object-oriented concept. It provides a highly optimized S4 class
implementation of various methods, including calibration by iterative
proportional fitting and simulated annealing, and modeling or data
fusion by logistic regression, regression tree methods and many other
methods.

\end{document}
