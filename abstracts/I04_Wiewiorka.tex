\providecommand{\main}{..} 
\documentclass[\main/boa.tex]{subfiles}

\begin{document}

\section{Big data genomic data warehouses analyses with R}

\begin{center}
  {\bf \index[a]{Wiewiórka Marek} Marek Wiewiórka$^{1^\star}$, \index[a]{Gambin Tomasz} Tomasz Gambin$^{1}$, \index[a]{Okoniewski Michał} Michał Okoniewski$^{2}$}
\end{center}

\vskip 0.3cm

\begin{affiliations}
\begin{enumerate}
\begin{minipage}{0.915\textwidth}
\centering
\item Institute of Computer Science, Warsaw University of Technology \\[-2pt]
\item ETH Zurich \\[-2pt]
\end{minipage}
\end{enumerate}
$^\star$Contact author: \href{mailto:marek.wiewiorka@gmail.com}{\nolinkurl{marek.wiewiorka@gmail.com}}\\
\end{affiliations}

\vskip 0.5cm

\begin{minipage}{0.915\textwidth}
\keywords big data; next-generation sequencing; data warehousing
\end{minipage}

\vskip 0.8cm


Genomic population studies incorporates storing, analyzing and interpretation of various kinds of genomic variants
as its central issue. When thousands of patients sequenced exomes and genomes are being sequenced, 
there is a growing need for efficient database storage systems, querying engines and powerful tools for statistical analyses.
Combination of \textbf{R} and scalable big data solutions such as Apache Impala, Apache Kudu, Apache Phoenix or Apache Kylin can address
many of the challenges in large scale genomic analyses.

Proof-of-concept of a big data solution for running analyses with R/RStudio will be presented during the talk.
Overview of some benchmark results, hints for designing and implementing scalable genomic data warehouse will be also covered
during the talk.
\end{document}
