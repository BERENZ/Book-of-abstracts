\providecommand{\main}{..} 
\documentclass[\main/boa.tex]{subfiles}

\begin{document}
\pagestyle{empty}

\section{Structural bioinformatician's notebooks with pdbeeR and knitr}

\begin{center}
  {\bf Paweł Piątkowski$^{1^\star}$}
\end{center}

\vskip 0.3cm

\begin{affiliations}
\begin{enumerate}
\begin{minipage}{0.915\textwidth}
\centering
\item International Institute of Molecular and Cell Biology \\[-2pt]
\end{minipage}
\end{enumerate}
$^\star$Contact author: \href{mailto:ppiatkowski@genesilico.pl}{\nolinkurl{ppiatkowski@genesilico.pl}}\\
\end{affiliations}

\vskip 0.5cm

\begin{minipage}{0.915\textwidth}
\keywords bioinformatics; biomolecules; knitr; reproducible research
\packages pdbeeR; knitr
\end{minipage}

\vskip 0.8cm

Working with biological molecules often involves much work with .pdb
files -- analyzing, subsetting, transforming and visualizing structural
data. Doing this by hand is tedious, hard to reproduce and prone to
errors. A new \textbf{R} package -- \emph{pdbeeR} -- can help you make
these chores fun and save your work as elegant knitr notebooks.

\end{document}
