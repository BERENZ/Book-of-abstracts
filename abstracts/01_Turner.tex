\providecommand{\main}{..} 
\documentclass[\main/boa.tex]{subfiles}

\begin{document}

\section{Addressing the Gender Gap in the R Project}

\begin{center}
  {\bf \index[a]{Heather Turner}$^{1^\star}$}
\end{center}

\vskip 0.3cm

\begin{affiliations}
\begin{enumerate}
\begin{minipage}{0.915\textwidth}
\centering
\item University of Warwick \\[-2pt]
\end{minipage}
\end{enumerate}
$^\star$Contact author: \href{mailto:ht@heatherturner.net}{\nolinkurl{ht@heatherturner.net}}\\
\end{affiliations}

\vskip 0.5cm

\begin{minipage}{0.915\textwidth}
\keywords R community; package development
\end{minipage}

\vskip 0.8cm

Despite \textbf{R}'s origins in the discipline of statistics and the
strong uptake of \textbf{R} in fields such as ecology and genomics,
where women are well represented in the workforce, the \textbf{R}
developer community looks more like that of computer science generally,
where women are in a distinct minority. If we consider contributions to
the \textbf{R} project, the situation is even worse. How can we
encourage more women to become developers and leaders in the \textbf{R}
community?

Earlier this year the \textbf{R} Foundation, a not-for-profit
organisation set up to support the \textbf{R} project
(\url{https://www.r-project.org/foundation/}), established a task force
to explore this question and to take actions to address the gender gap
(\url{http://forwards.github.io/};
\url{https://twitter.com/RWomenTaskforce}). In this talk I will give an
overview of the activities of the taskforce so far and our plans for the
future. I will also share some tips for women looking to get more
involved and ideas of ways everyone can help to make our community more
inclusive.

\end{document}
