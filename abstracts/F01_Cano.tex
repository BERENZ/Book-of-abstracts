\providecommand{\main}{..} 
\documentclass[\main/boa.tex]{subfiles}

\begin{document}

\section{Unattended SVM parameters fitting for monitoring nonlinear profiles}

\begin{center}
  {\bf Emilio L. Cano$^{1^\star}$, Javier M. Moguerza$^{2}$, Mariano Prieto Corcoba$^{3}$}
\end{center}

\vskip 0.3cm

\begin{affiliations}
\begin{enumerate}
\begin{minipage}{0.915\textwidth}
\centering
\item The University of Castilla-La Mancha \\[-2pt]
\item Rey Juan Carlos University \\[-2pt]
\item ENUSA Industrias Avanzadas \\[-2pt]
\end{minipage}
\end{enumerate}
$^\star$Contact author: \href{mailto:emilio@lcano.com}{\nolinkurl{emilio@lcano.com}}\\
\end{affiliations}

\vskip 0.5cm

\begin{minipage}{0.915\textwidth}
\keywords quality control; SVMs; nonlinear profiles
\packages SixSigma; e1071; qcc
\end{minipage}

\vskip 0.8cm

The monitoring of nonlinear profiles is a recent quality control
technique. It allows to apply Statistical Process Control (SPC) methods
to processes in which, rather than having a quality characteristic,
there is a sort of nonlinear function that characterises the process.
This method has been implemented in the \emph{SixSigma} \textbf{R}
package. The underlying idea is to compute a prototype profile and
confidence bands using a data set from an in-control process, monitoring
subsequent profiles thereafter. Thus, the same methodology used in
well-known Shewhart control charts can be applied to complex processes.
To this aim, raw data can be used. Nevertheless, using regularisation
theory nonlinear profiles can be smoothed in order to better represent
and analyse the profiles. In this work, we use Support Vector Machines
(SVMs) to smooth profiles throughout the control process. Consequently,
SVM parameters must be selected in order to reach a good fit of the
nonlinear function at hand. Such parameters, namely: \(C\) and
\(\epsilon\), can be explicitely assigned in the \emph{smoothProfile()}
function of the \emph{SixSigma} package. However, a quality control
practicioner seldom knows about SVMs, needless to say that they have no
time to spend modelling functions. Hence, we rely on to automatically
fit the SVM parameters using the process data, thereby achieving
unattended SVM fitting. Furthermore, noise is previously estimated by
means of a loess fit.

\end{document}
