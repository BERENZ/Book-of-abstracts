\providecommand{\main}{..} 
\documentclass[\main/boa.tex]{subfiles}

\begin{document}

\section{Using R for artistic purposes}

\begin{center}
  {\bf Päivi Julin$^{1^\star}$}
\end{center}

\vskip 0.3cm

\begin{affiliations}
\begin{enumerate}
\begin{minipage}{0.915\textwidth}
\centering
\item Aureolis \\[-2pt]
\end{minipage}
\end{enumerate}
$^\star$Contact author: \href{mailto:julin@apup.org}{\nolinkurl{julin@apup.org}}\\
\end{affiliations}

\vskip 0.5cm

\begin{minipage}{0.915\textwidth}
\keywords visualization; design; typography; alternative; Processing
\packages ggplot2; grid; Rserve; RColorBrewer
\end{minipage}

\vskip 0.8cm

\textbf{R} can be tamed for serious business usage and data analysis,
but this tool also has abilities for a different kind of approach:
creating data art. Even \textbf{R} might not be mainly marketed with its
graphics; it does have competence in genuine creativeness for unusual
visualizations. Thus, \textbf{R} is much more than just bar plots,
charts and maps -- it is only a matter of practice and adaptation.

Data art focuses using raw data to produce visualizations with both
aesthetic and informative scope. Having a creative process with digital
media, real-time data and \textbf{R}, may lead to producing alternative
interpretations from traditional information. Playing with \textbf{R}
graphics devices, one is able to adjust any kind of visual presentations
inside \textbf{R} console. Another option is to handle data manipulation
with \textbf{R} and print output with alternative graphic portals - such
as Processing or D3.

The presentation involves graphical visualizations, technical solutions,
and general ideas, how to use \textbf{R} in alternative ways. Telling a
story from data is a fun journey, therefore playing with \textbf{R} can
be rather an amusing \textbf{R}-experience. This program being my muse,
the aim is to inspire others.

\end{document}
