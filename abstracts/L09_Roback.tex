\providecommand{\main}{..} 
\documentclass[\main/boa.tex]{subfiles}

\begin{document}

\section{Using R to incorporate data science into the undergraduate statistics
curriculum}

\begin{center}
  {\bf \index[a]{Roback Paul} Paul Roback$^{1^\star}$}
\end{center}

\vskip 0.3cm

\begin{affiliations}
\begin{enumerate}
\begin{minipage}{0.915\textwidth}
\centering
\item St.~Olaf College \\[-2pt]
\end{minipage}
\end{enumerate}
$^\star$Contact author: \href{mailto:roback@stolaf.edu}{\nolinkurl{roback@stolaf.edu}}\\
\end{affiliations}

\vskip 0.5cm

\begin{minipage}{0.915\textwidth}
\keywords undergraduate curriculum; statistical computing; data science
\packages \index[p]{ggplot2} ggplot2; \index[p]{dplyr} dplyr; \index[p]{ISLR} ISLR
\end{minipage}

\vskip 0.8cm

The American Statistical Association recently endorsed an updated set of
Curriculum Guidelines for Undergraduate Programs in Statistical Science,
and the first key point is: `Increased importance of data science'. In
addition to traditional topics in statistical reasoning, modeling, and
inference, today's future statisticians also need the ability to access
and manipulate data and to engage in algorithmic problem-solving to
maximize their contributions. In this talk, I will explore what the
increased importance of data science means for undergraduate programs in
statistics and describe what we have done using \textbf{R} at St.~Olaf
College, an undergraduate institution of 3000 students in the United
States. For example, statisticians partnered with colleagues in computer
science to develop a `statistics-infused' version of introductory
computer science (CS125). This new statistical computing course
introduces students to key foundational ideas in computer science by
using mostly \textbf{R} (and some Python) and motivating the ideas
through a data analysis context. Students learn about functions, loops,
matrices, text strings, data scraping, and SQL while also discovering
data visualization and classification methods. In fact, this course is
one of seven featured in The American Statistician article `Data science
in the statistics curricula: preparing students to 'think with data''.
In addition to CS125, we have integrated more data wrangling and data
visualization into our regression course, added a course on Algorithms
for Decision Making which primarily explores classification methods
using \textbf{R}, and developed an entire course on Data Visualization
with \textbf{R} to pilot this fall.

\end{document}
