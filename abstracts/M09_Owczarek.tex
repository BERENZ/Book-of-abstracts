\providecommand{\main}{..} 
\documentclass[\main/boa.tex]{subfiles}

\begin{document}

\section{An R implementation of Kauffman's NK model}

\begin{center}
  {\bf Tomasz Owczarek$^{1^\star}$}
\end{center}

\vskip 0.3cm

\begin{affiliations}
\begin{enumerate}
\begin{minipage}{0.915\textwidth}
\centering
\item Faculty of Organization and Management, Silesian University of
Technology \\[-2pt]
\end{minipage}
\end{enumerate}
$^\star$Contact author: \href{mailto:tomasz.owczarek@polsl.pl}{\nolinkurl{tomasz.owczarek@polsl.pl}}\\
\end{affiliations}

\vskip 0.5cm

\begin{minipage}{0.915\textwidth}
\keywords nk model; complexity; fitness landscape; agent-based modeling
\packages dplyr; ggplot2; igraph
\end{minipage}

\vskip 0.8cm

Stuart Kauffman's NK model is a simple but powerful model of complex
system with many interactions between its elements. It generates so
called fitness landscape, i.e.~a space consisted of many points with
different attractiveness (fitness). The great advantage of the model is
that only two parameters (\(N\) and \(K\)) are required to be controlled
to generate environments with `tunable complexity' -- from very simple
with one local and global optimum to completely random, with a large
number of local optima. The NK model is used in the studies of many
phenomena in various fields e.g.~biology, economics, organizational
studies or engineering. In my presentation I would like to introduce an
\textbf{R} package which implements the NK model and show how to use it
to generate fitness landscapes and explore their properties for
different influence matrices.

\end{document}
