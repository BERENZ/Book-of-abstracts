\providecommand{\main}{..} 
\documentclass[\main/boa.tex]{subfiles}

\begin{document}
\pagestyle{empty}

\section{Workflow around modelling in Data Science / R}

\begin{center}
  {\bf Filip Stachura$^{1^\star}$, Olga Mierzwa$^{1}$, Paweł Przytuła$^{1}$}
\end{center}

\vskip 0.3cm

\begin{affiliations}
\begin{enumerate}
\begin{minipage}{0.915\textwidth}
\centering
\item Appsilon \\[-2pt]
\end{minipage}
\end{enumerate}
$^\star$Contact author: \href{mailto:filip@appsilon.pl}{\nolinkurl{filip@appsilon.pl}}\\
\end{affiliations}

\vskip 0.5cm

\begin{minipage}{0.915\textwidth}
\keywords workflow; process; data science
\packages dplyr; caret; randomForest
\end{minipage}

\vskip 0.8cm

Working as a data scientist usually means working with data, building
models, evaluating the results, translating them into actionable
insights, getting feedback from experts and repeating the process.
Hardly ever one starts with the model that finally will be used in
production. It is usually trial and error process of trying new things
and experimenting. That's why data science project could get
disorganized rapidly. Every test involves a new script, each script
requires a multiple arguments and produces one or more data files.
Keeping track of all this implied structure is a pain.

The talk stresses the importance of having a process around data science
related tasks, while keeping the main focus on creating a data product.
We demonstrate the implementation of the light data science workflow --
Dataflows. Dataflows allows \textbf{R} users to create pipelines,
without writing extra code, is self documenting and easy to start
working with.

During the talk we show the benefits of formalizing and structuring the
process of model building in \textbf{R}. We mention the bottlenecks,
propose solutions and biggest wins accomplished by introducing
Dataflows. We strive to share our hands-on experience from various data
science projects using \textbf{R}.

Target audience practitioners, data scientists and researchers
interested in rising standards of their current data product creation
process, with the stress on reproducibility, early error detection, easy
of results evaluation, comparison and communication with less technical
colleagues.

\end{document}
