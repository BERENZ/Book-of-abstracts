\providecommand{\main}{..} 
\documentclass[\main/boa.tex]{subfiles}

\begin{document}
\pagestyle{empty}

\section{Reporting automation with ReporteRs}

\begin{center}
  {\bf David Gohel$^{1^\star}$}
\end{center}

\vskip 0.3cm

\begin{affiliations}
\begin{enumerate}
\begin{minipage}{0.915\textwidth}
\centering
\item ArData \\[-2pt]
\end{minipage}
\end{enumerate}
$^\star$Contact author: \href{mailto:david.gohel@ardata.fr}{\nolinkurl{david.gohel@ardata.fr}}\\
\end{affiliations}

\vskip 0.5cm

\begin{minipage}{0.915\textwidth}
\keywords reporting; Word; PowerPoint; tables; graphics
\packages ReporteRs; rtable; shiny
\end{minipage}

\vskip 0.8cm

\textbf{R} usage has risen in companies and reporting capabilities of
\textbf{R} is now an important subject. The way results are compiled in
documents and spread to colleagues or customers can be time consuming.
Key points are reproducibility, ability to produce pretty outputs and
document types.

Microsoft document formats are still pretty ubiquitous in corporate
environments. The package \emph{ReporteRs} make easier the production of
these documents on any platform. \textbf{R} users can produce pretty
formatted outputs into Word or PowerPoint documents with only \textbf{R}
code. It comes with an API to produce advanced graphical and tabular
reporting.

In this talk, I will introduce \emph{ReporteRs} and others related
packages. Major features will be explained and illustrated. A use case
will be presented to show a clinical reporting application made with
\emph{shiny}.

\end{document}
