\providecommand{\main}{..} 
\documentclass[\main/boa.tex]{subfiles}

\begin{document}

\section{Aargh I have to teach R (Experiences in the teaching of R)}

\begin{center}
  {\bf Martin Schneider$^{1^\star}$}
\end{center}

\vskip 0.3cm

\begin{affiliations}
\begin{enumerate}
\begin{minipage}{0.915\textwidth}
\centering
\item eoda GmbH \\[-2pt]
\end{minipage}
\end{enumerate}
$^\star$Contact author: \href{mailto:martin.schneider@eoda.de}{\nolinkurl{martin.schneider@eoda.de}}\\
\end{affiliations}

\vskip 0.5cm

\begin{minipage}{0.915\textwidth}
\keywords teaching; training
\packages shiny
\end{minipage}

\vskip 0.8cm

As \textbf{R} is gaining popularity, so there is the interest in
\textbf{R} courses. \textbf{R}, generally considered as a programming
language with a steep learning curve, has surely its own typical
pitfalls. The aim of this talk is to give an overview of the general
do's and don'ts regarding the teaching of \textbf{R} and how a typical
\textbf{R} course can be designed. Design and execution of a course
depend on a variety of influences such as different types of course
participants, content and context. Course participants usually differ
regarding prior knowledge, interest in the material, ambitions as well
as objectives to be reached. Generally, \textbf{R} courses focus either
more on statistical or on programming topics. Another crucial aspect is
whether the participants come from an academic or an industrial
background, in other words, the context in which the course takes place.
The talk is based on the experiences of an \textbf{R} trainer teaching
\textbf{R} to academics, as well as in companies, with a topic area
ranging from teaching the fundamentals to specialised workshops
(e.g.~performing a Network Analysis in a \emph{shiny} app).

\end{document}
