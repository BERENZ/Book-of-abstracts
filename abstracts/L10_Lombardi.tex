\providecommand{\main}{..} 
\documentclass[\main/boa.tex]{subfiles}

\begin{document}

\section{Analysing the statistical effects of manipulated data}

\begin{center}
  {\bf \index[a]{Lombardi Luigi} Luigi Lombardi$^{1^\star}$, \index[a]{Bressan Marco} Marco Bressan$^{1}$}
\end{center}

\vskip 0.3cm

\begin{affiliations}
\begin{enumerate}
\begin{minipage}{0.915\textwidth}
\centering
\item Department of Psychology and Cognitive Science, University of Trento \\[-2pt]
\end{minipage}
\end{enumerate}
$^\star$Contact author: \href{mailto:luigi.lombardi@unitn.it}{\nolinkurl{luigi.lombardi@unitn.it}}\\
\end{affiliations}

\vskip 0.5cm

\begin{minipage}{0.915\textwidth}
\keywords manipulated data; probabilistic models; generative models, sample
generation by replacement
\packages \index[p]{sgr} sgr
\end{minipage}

\vskip 0.8cm

Many self-report measures collected in different research fields such
as, for example, social and psychological studies, marketing and
epidemiological studies may be affected by manipulated information by
respondents. For example, an individual may deliberately attempt to
manipulate or distort responses to simulate grossly exaggerated
psychiatric symptoms in order to obtaining financial compensation or
avoiding being charged with a crime. However, in other situations data
manipulation can arise without a clear intention or goal by the
respondent, for example, when an answer to a question is given under
time pressure, or in heavily stressing contexts (e.g.~during a job
interview). Despite the reasons and motivations that induce people to
(voluntarily or involuntarily) manipulate data can be different across
contexts and situations, we believe that these processes can be
represented by a single but flexible probabilistic model. In this
contribution, we introduce an \textbf{R} package, called \emph{sgr},
that can be used to perform data analysis on manipulated data according
to a sample generation by replacement approach. The package includes
functions for making simple inferences about discrete, ordinal as well
as categorical data under one or more scenarios of data manipulation.
The package also allows to quantify uncertainty in inferences based on
possible manipulated data as well as to study the implications of
manipulated data for empirical results.

\end{document}
