\providecommand{\main}{..} 
\documentclass[\main/boa.tex]{subfiles}

\begin{document}

\section{Machine learning modeling of phenological phases in Poland}

\begin{center}
  {\bf \index[a]{Czernecki Bartosz} Bartosz Czernecki$^{1^\star}$, \index[a]{Nowosad Jakub} Jakub Nowosad$^{2}$, \index[a]{Jabłońska Katarzyna} Katarzyna Jabłońska$^{3}$}
\end{center}

\vskip 0.3cm

\begin{affiliations}
\begin{enumerate}
\begin{minipage}{0.915\textwidth}
\centering
\item Adam Mickiewicz University in Poznań, Department of Climatology \\[-2pt]
\item Adam Mickiewicz University, Department of Geoinformation \\[-2pt]
\item Institute of Meteorology and Water Management - National Research
Institute \\[-2pt]
\end{minipage}
\end{enumerate}
$^\star$Contact author: \href{mailto:nwp@amu.edu.pl}{\nolinkurl{nwp@amu.edu.pl}}\\
\end{affiliations}

\vskip 0.5cm

\begin{minipage}{0.915\textwidth}
\keywords phenology; machine learning; modeling; satellite; meteorology;
climatology
\packages \index[p]{caret}; \index[p]{raster}; \index[p]{Boruta}; \index[p]{parallel}
\end{minipage}

\vskip 0.8cm

Changes in timing of phenological phases are important proxies in
contemporary climate research. The aim of the study was to create and
evaluate different statistical models for reconstructing and predicting
the selected phenological phase within \emph{caret} \textbf{R} package.
Three types of data sources were applied as predictors:

\begin{enumerate}
\def\labelenumi{\roman{enumi}.}
\tightlist
\item
  satellite derived products,
\item
  preprocessed gridded meteorological data
\item
  spatial features (longitude, latitude, altitude) of the monitoring
  sites.
\end{enumerate}

The obtained results has shown potential for coupling meteorological
derived indices with remote sensing products in terms of phenological
(late spring) modelling. It was also shown that choosing a specific set
of predictors and applying a robust preprocessing procedures is more
affecting final results than applying a statistical model.

\end{document}
