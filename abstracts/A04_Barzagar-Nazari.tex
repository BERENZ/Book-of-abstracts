\providecommand{\main}{..} 
\documentclass[\main/boa.tex]{subfiles}

\begin{document}

\section{Data science outside the box: Developing a generic scoring algorithm for
customer acquisition}

\begin{center}
  {\bf Erik Barzagar-Nazari$^{1^\star}$}
\end{center}

\vskip 0.3cm

\begin{affiliations}
\begin{enumerate}
\begin{minipage}{0.915\textwidth}
\centering
\item eoda GmbH \\[-2pt]
\end{minipage}
\end{enumerate}
$^\star$Contact author: \href{mailto:erik.barzagar-nazari@eoda.de}{\nolinkurl{erik.barzagar-nazari@eoda.de}}\\
\end{affiliations}

\vskip 0.5cm

\begin{minipage}{0.915\textwidth}
\keywords data mining; customer acquisition; case study
\end{minipage}

\vskip 0.8cm

One major task in virtually every predictive modelling project is to
find the method best suited for the problem on hand. Luckily, most of
the time data scientists can rely on one of the many already existing
and well established methods such as Random Forests, Gradient Boosting
Machines, Neural Networks or Support Vector Machines to solve a variety
of regression and classification problems. However, in some cases these
standard approaches are not directly applicable to the problem on hand
and data scientists need to become creative. In this talk, we will
present a case study about a project we recently conducted for the
databyte GmbH in Germany. As one of the leading business information
providers, databyte maintains a vast database of several million
companies, containing information such as revenue, size and many other
properties. The aim of the project was to develop an application which,
after provided with the customer base of a databyte client, would be
able to score companies in the databyte database in order to identify
the most promising contacts for direct marketing campaigns. While
developing this application, we were facing two major requirements:
first of all, each client has its own customer base, hence we could not
just train one model; in fact, the algorithm must be able to `train
itself' in every run. Secondly, the customer base only contains
`positive data', that means we are dealing with a so-called
positive-unlabelled-problem.

\end{document}
