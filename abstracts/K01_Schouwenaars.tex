\providecommand{\main}{..} 
\documentclass[\main/boa.tex]{subfiles}

\begin{document}

\section{Revolutionize how you teach and blog: add interactivity}

\begin{center}
  {\bf \index[a]{Filip Schouwenaars}$^{1^\star}$}
\end{center}

\vskip 0.3cm

\begin{affiliations}
\begin{enumerate}
\begin{minipage}{0.915\textwidth}
\centering
\item DataCamp \\[-2pt]
\end{minipage}
\end{enumerate}
$^\star$Contact author: \href{mailto:filip@datacamp.com}{\nolinkurl{filip@datacamp.com}}\\
\end{affiliations}

\vskip 0.5cm

\begin{minipage}{0.915\textwidth}
\keywords education; web technology; reporting; interactivity
\packages tutorial; knitr; rmarkdown
\end{minipage}

\vskip 0.8cm

\textbf{R} vignettes, blog posts and teaching materials are typically
standard web pages generated with \emph{rmarkdown}. DataCamp has
developed a framework to make this static content interactive:
\textbf{R} code chunks are converted into an \textbf{R}-session backed
editor so readers can experiment. This talk will explain the inner
workings of the technology, as well as a the tutorial \textbf{R} package
that makes the transition to interactive web pages seamless. Some
hands-on examples will showcase the remarkable ease with which you can
convert \textbf{R} Markdown documents, vignettes and Jekyll-powered
blogs into interactive \textbf{R} playgrounds.

\end{document}
