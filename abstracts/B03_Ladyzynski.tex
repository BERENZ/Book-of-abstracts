\providecommand{\main}{..} 
\documentclass[\main/boa.tex]{subfiles}

\begin{document}

\section{R tools and tricks for marketing inference in a big internet company}

\begin{center}
  {\bf Paweł Ładyżyński$^{1^\star}$}
\end{center}

\vskip 0.3cm

\begin{affiliations}
\begin{enumerate}
\begin{minipage}{0.915\textwidth}
\centering
\item Naspers Classifieds \\[-2pt]
\end{minipage}
\end{enumerate}
$^\star$Contact author: \href{mailto:pawelladyz@wp.pl}{\nolinkurl{pawelladyz@wp.pl}}\\
\end{affiliations}

\vskip 0.5cm

\begin{minipage}{0.915\textwidth}
\keywords marketing inference; clustering; big data; predictive modeling; feature
engineering; feature selection
\packages data.table; dplyr; ggplot; h2o; rmarkdown; shiny
\end{minipage}

\vskip 0.8cm

Data is becoming an integral part of digital marketing, as businesses
realize the power of information to create successful campaigns and see
real-time results. In 2016, big data continues its growth as an
important part of supporting business decisions. Armed with information
on customer behaviors and purchases, we are now able to build profiles
or user personas which ensure each marketing effort is geared toward a
specific type of customer. However, the crucial part of business
reasoning is to examine large data sets to uncover hidden patterns,
unknown correlations, market trends, customer preferences and other
useful business information. This step is usually connected with looking
for new libraries or packages which performance is good enough to deal
with large amounts of data. Fortunately, there are many \textbf{R}
packages, like \emph{data.table} or \emph{h2o}, which enables \textbf{R}
to be involved in the big data reasoning process.

\end{document}
