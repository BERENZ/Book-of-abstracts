\providecommand{\main}{..} 
\documentclass[\main/boa.tex]{subfiles}

\begin{document}

\section{Browse Till You Die: Scalable Hierarchical Bayesian Modeling of cookie
deletion}

\begin{center}
  {\bf Jakub Glinka$^{1^\star}$}
\end{center}

\vskip 0.3cm

\begin{affiliations}
\begin{enumerate}
\begin{minipage}{0.915\textwidth}
\centering
\item GfK SE, Nuremberg \\[-2pt]
\end{minipage}
\end{enumerate}
$^\star$Contact author: \href{mailto:Jakub.Glinka@gfk.com}{\nolinkurl{Jakub.Glinka@gfk.com}}\\
\end{affiliations}

\vskip 0.5cm

\begin{minipage}{0.915\textwidth}
\keywords hierarchical bayesian modeling; model based machine learning; consensus
MCMC
\packages rstan; parallelMCMCcombine; Rcpp
\end{minipage}

\vskip 0.8cm

The common approach for tracking the device's on-line movement is
through cookies - small portion of information stored within user
Internet Browser. This enables Market Research companies to assign web
browsing data to one specific browser. The main problem within cookie
tracking framework is to assess whether on given day the lack of its
activity is due to the real absence or deletion. In usual site-centric
approach one can only observe cookie's digital footprint on limited
number of media providers which leads to the highly skewed data,
moreover the deletion moment is not directly observable. In order to
deal with mentioned challenges we designed Hierarchical Bayes model of
the cookie behavior that enables us to pose questions about the
probability of the cookie deletion. We will present how it fits Model
Based Machine Learning Paradigm and how one can effciently estimate
model coefficients using existing \textbf{R} packages.

\end{document}
