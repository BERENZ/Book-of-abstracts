\providecommand{\main}{..} 
\documentclass[\main/boa.tex]{subfiles}

\begin{document}

\section{Applying genetic algorithms to calibrate a processing chain for a
Landsat-based time series analysis of disturbance - regrowth dynamics in
tropical forests}

\begin{center}
  {\bf \index[a]{Fabián Santos}$^{1^\star}$, \index[a]{Gunther Menz}$^{1}$, \index[a]{Olena Dubvyky}$^{1}$}
\end{center}

\vskip 0.3cm

\begin{affiliations}
\begin{enumerate}
\begin{minipage}{0.915\textwidth}
\centering
\item University of Bonn \\[-2pt]
\end{minipage}
\end{enumerate}
$^\star$Contact author: \href{mailto:fabian_santos_@hotmail.com}{\nolinkurl{fabian\_santos\_@hotmail.com}}\\
\end{affiliations}

\vskip 0.5cm

\begin{minipage}{0.915\textwidth}
\keywords time-series analysis; genetic algorithms; Landsat; tropical forests
monitoring
\packages GA; changepoint; BreakoutDetection; ecp
\end{minipage}

\vskip 0.8cm

We present an innovative approach to calibrate processing chains
designed for a time-series analysis based on genetic algorithms (GA).
Using a case study of disturbance-regrowth monitoring employing Landsat
data in tropical forests located in the Amazonian region of Ecuador.
This area is characterized by mountainous terrain, high topographic
relief and extensive cloud cover. These conditions can cause noise in
the time-series of Landsat data and require different corrections before
its analysis. We describe here the use of \emph{GA} to reduce
calibration uncertainties, enhance the output accuracy, and avoid
unnecessary processing caused by trial-and-error approaches in remote
sensing.

\end{document}
