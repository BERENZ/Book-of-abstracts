\providecommand{\main}{..} 
\documentclass[\main/boa.tex]{subfiles}

\begin{document}

\section{The full process of creating an R application in recommendation systems:
from Dockerfile to Zabbix monitoring}

\begin{center}
  {\bf Natalia Potocka$^{1^\star}$}
\end{center}

\vskip 0.3cm

\begin{affiliations}
\begin{enumerate}
\begin{minipage}{0.915\textwidth}
\centering
\item Grupa Wirtualna Polska \\[-2pt]
\end{minipage}
\end{enumerate}
$^\star$Contact author: \href{mailto:Natalia.Potocka@grupawp.pl}{\nolinkurl{Natalia.Potocka@grupawp.pl}}\\
\end{affiliations}

\vskip 0.5cm

\begin{minipage}{0.915\textwidth}
\keywords R application; docker; monitoring
\packages rkafka; elastic; httr; RJBDC; RZabbix
\end{minipage}

\vskip 0.8cm

The presentation covers the main steps of creating a reproducible
\textbf{R} application. First of all, there is a need of collecting the
data. We present how to receive it from various sources, including
Kafka, Elasticsearch or Hadoop (with the help of suitable \textbf{R}
packages). After transforming the data, doing the necessary calculations
and obtaining the desired outcome we show how to push out the results.
When the application is ready and running, it needs to be controlled. We
present the proccess of creating reliable monitoring (with the help of
Zabbix and Hipchat notifications). All of that can be wrapped up in a
Docker to make the application reproducible in different environments
and platforms.

\end{document}
