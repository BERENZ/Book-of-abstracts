\providecommand{\main}{..} 
\documentclass[\main/boa.tex]{subfiles}

\begin{document}

\section{What are sampling errors in the vegetation studies using visual
estimation of presence and cover of plants? R can help}

\begin{center}
  {\bf Damian Chmura$^{1^\star}$, Anna Salachna$^{1}$, Edyta Sierka$^{2}$}
\end{center}

\vskip 0.3cm

\begin{affiliations}
\begin{enumerate}
\begin{minipage}{0.915\textwidth}
\centering
\item University of Bielsko-Biala \\[-2pt]
\item University of Silesia \\[-2pt]
\end{minipage}
\end{enumerate}
$^\star$Contact author: \href{mailto:dchmura@ath.bielsko.pl}{\nolinkurl{dchmura@ath.bielsko.pl}}\\
\end{affiliations}

\vskip 0.5cm

\begin{minipage}{0.915\textwidth}
\keywords repeatability; interrater reliability, agreement; intraclass correlation
coefficient
\packages nlme; multilevel; irr; betapart; vegan
\end{minipage}

\vskip 0.8cm

In vegetation science (e.g.~phytosociology), visual estimates of plant
cover belong to the most frequently used descriptors. The main reason
for their attractiveness lies in the very low cost of the data obtained
in this way, both in terms of labour, time and equipment. However, they
are a subjective methods which result in sampling error, difficult to
control. It is emphasised that using the visible estimation of cover
requires some attempt and experience. However, many studies showed that
comparison between observers yielded differences in the cover of
recorded species as well as between repeated estimates of the same
observer. We conducted several experiments with visual measurements of
tree canopy, i.e.~cover of tree layer and cover of herb layer in forest
habitats. The experiments were performed with various raters, differing
in experience in fieldwork: from students to professional researchers.
Several field methods were applied: canopyscope, Braun-Blanquet
approach, different scales of estimates; point method and a few
vegetation types were chosen. Contrary to similar studies, where usually
descriptive statistics are shown, we employed different statistical
methods e.g.~intra-class correlations, analyses of species
pseudoturnover and nestedness -- associated with loss of information
among observes. The selected in \textbf{R} statistical methods turned
out to be sensitive and robust tools. They revealed that even between
professional researchers distinct differences appear. Advantages and
disadvantages of available field methods are discussed and some possible
improvements are suggested.

\end{document}
