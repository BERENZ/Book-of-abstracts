\providecommand{\main}{..} 
\documentclass[\main/boa.tex]{subfiles}

\begin{document}

\section{R for pharmacokineticists - smulation of steady-state concentrations of
amiodarone in heart compartmental model as an example.}

\begin{center}
  {\bf \index[a]{Zofia Tylutki}$^{1^\star}$, \index[a]{Sebastian Polak}$^{1}$}
\end{center}

\vskip 0.3cm

\begin{affiliations}
\begin{enumerate}
\begin{minipage}{0.915\textwidth}
\centering
\item Faculty of Pharmacy, Jagiellonian University \\[-2pt]
\end{minipage}
\end{enumerate}
$^\star$Contact author: \href{mailto:zofia.tylutki@doctoral.uj.edu.pl}{\nolinkurl{zofia.tylutki@doctoral.uj.edu.pl}}\\
\end{affiliations}

\vskip 0.5cm

\begin{minipage}{0.915\textwidth}
\keywords pharmacokinetics; modelling; parameter optimization
\packages deSolve; FME
\end{minipage}

\vskip 0.8cm

Amiodarone poses a known risk of torsade de pointes arrhythmia
induction, according to CredibleMeds classification. Thus, the data on
its effective concentration at heart tissue are desired. Patients
prescribed to amiodarone usually are on long-term treatment following
the standard dosing regimen, i.e.~200 mg t.i.d. as a priming dose, and
100 mg q.d. as a sustaining dose. The aim of the study was to build a
functional heart model in order to simulate the amiodarone cardiac
concentration at steady state. Three compartments of physiological
volumes of plasma (central compartment), cardiac tissue, and pericardial
fluid linked via first-order rate constants represented model structure.
The model was described by set of differential equations, and written in
\textbf{R} v.3.1.3. Numerical solutions were computed using
\emph{deSolve} library. The model parameters were optimized using
\emph{FME} package. The fit was performed to mean amiodarone
concentrations observed by Escoubet et al. in plasma and heart tissue,
respectively, in 61 patients given amiodarone in a single dose. The
optimized parameters ({[}h-1{]} ka: 0.038, kht\_in: 0.200, kht\_out:
0.149, kpf\_in: 0.680, kpf\_out: 9.700, ke: 5.000) were used to simulate
amiodarone time-concentration profiles in two-months standard dosage
schedule of amiodarone. An event function was describing multiple
administrations. The simulated cardiac concentrations at steady-state
were ca. 20 times higher than corresponding amiodarone levels in central
compartment, which occurred to be in accordance with the ratios reported
in literature that can be assumed to refer to steady state situation.
Results support the feasibility of the model as well as the approach to
simulate steady-state.

\end{document}
