\providecommand{\main}{..} 
\documentclass[\main/boa.tex]{subfiles}

\begin{document}

\section{k Prototypes Clustering of Mixed Type Data}

\begin{center}
  {\bf \index[a]{Gero Szepannek}$^{1^\star}$}
\end{center}

\vskip 0.3cm

\begin{affiliations}
\begin{enumerate}
\begin{minipage}{0.915\textwidth}
\centering
\item Stralsund University of Applied Sciences \\[-2pt]
\end{minipage}
\end{enumerate}
$^\star$Contact author: \href{mailto:gero.szepannek@fh-stralsund.de}{\nolinkurl{gero.szepannek@fh-stralsund.de}}\\
\end{affiliations}

\vskip 0.5cm

\begin{minipage}{0.915\textwidth}
\keywords clustering; data mining; classification; k prototypes; multivariate
statistics
\packages clustMixType; klaR; cluster; fpc
\end{minipage}

\vskip 0.8cm

Most literature in mutlivariate statistics deals with numeric data, same
for clustering. In practical applications the analyst is often
confronted with mixed type data of both numeric and factor variables.
Hierarchical clustering can easily be extended to categorical data by
specification of an apropriate (dis)similarity measure (e.g.~Gower
distance). The k means algorithm relies on euclidean distance. The
\emph{klaR} package offers an \textbf{R} implementation of the k modes
extension to the k means algorithm for categorical data. Further, the
recent \emph{clustMixType} package also makes the k prototypes
generalization to mixed type data acessible to \textbf{R} users. Both
algorithms are presented together with their usage in \textbf{R}.

\end{document}
