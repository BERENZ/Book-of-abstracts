\providecommand{\main}{..} 
\documentclass[\main/boa.tex]{subfiles}

\begin{document}

\section{Discrete Choice Models in R}

\begin{center}
  {\bf \index[a]{Guhl Daniel} Daniel Guhl$^{1^\star}$, \index[a]{Gabel Sebastian} Sebastian Gabel$^{1}$}
\end{center}

\vskip 0.3cm

\begin{affiliations}
\begin{enumerate}
\begin{minipage}{0.915\textwidth}
\centering
\item Humboldt University, Berlin \\[-2pt]
\end{minipage}
\end{enumerate}
$^\star$Contact author: \href{mailto:daniel.guhl@hu-berlin.de}{\nolinkurl{daniel.guhl@hu-berlin.de}}\\
\end{affiliations}

\vskip 0.5cm

\begin{minipage}{0.915\textwidth}
\keywords discrete choice models; econometrics; Bayesian statistics; Stan
\packages \index[p]{rstan} rstan; \index[p]{mlogit} mlogit; \index[p]{gmnl} gmnl; \index[p]{bayesm} bayesm; \index[p]{ChoiceModelR}  ChoiceModelR
\end{minipage}

\vskip 0.8cm

Discrete choice models (DCM) are a widely used class of models in
economics, marketing, and transportation science. These models are
rooted in random utility theory and can be applied if a decision maker
picks one alternative out of multiple options. The most commonly used
variations of DCM are the multinomial logit model (MNL), the mixed logit
model (MXL), and multinomial probit (MNP). Many \textbf{R} packages
(e.\,g., \emph{mlogit}, \emph{gmnl}, \emph{bayesm}) are available for
frequentist and Bayesian inference of DCM. However, the different
packages lack a unified data interface, common structure of functions
and methods, and comparable output. Therefore, the analyst is faced with
a typical dilemma: to test several models and follow established
research processes (and maybe fulfill reviewer requests) one has to deal
with data transformations, model translations, and output formatting,
which is tedious, time consuming, and error-prone. In addition, the
established \textbf{R} packages lack flexibility and transparency. To
this end, we propose to use Stan, a general purpose modeling language
for Bayesian inference written in C++ with interfaces to \textbf{R},
python, Matlab, Julia and Stata. We compare Stan-implementations of 
MXL models using simulated and real data. We show how Stan can be
used for estimating more sophisticated models (e.\,g., models in
willingness-to-pay space). We also provide a quick reference over our
\textbf{R} package \emph{dcm} that aims at closing the gap for existing
packages regarding a unified data API and comparable model output.

\end{document}
