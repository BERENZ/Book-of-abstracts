\documentclass[11pt, a4paper]{article}
\usepackage[utf8]{inputenc}
\usepackage[T1]{fontenc}
\usepackage{eurosym}
\usepackage{amsfonts, amsmath, hanging, hyperref, parskip, times}
\usepackage[numbers]{natbib}
\usepackage[pdftex]{graphicx}
\hypersetup{
  colorlinks,
  linkcolor=black,
  urlcolor=black,
  citecolor=black
}

\let\section=\subsubsection
\newcommand{\pkg}[1]{{\normalfont\fontseries{b}\selectfont #1}}
\let\proglang=\textit
\let\code=\texttt
\renewcommand{\title}[1]{\begin{center}{\bf \LARGE #1}\end{center}}
\newcommand{\affiliations}{\footnotesize\centering}
\newcommand{\keywords}{\paragraph{Keywords:}}
\newcommand{\packages}{\paragraph{R packages:}}

\providecommand{\tightlist}{%
  \setlength{\itemsep}{0pt}\setlength{\parskip}{0pt}}

\setlength{\topmargin}{-15mm}
\setlength{\oddsidemargin}{-2mm}
\setlength{\textwidth}{165mm}
\setlength{\textheight}{250mm}


\begin{document}
\pagestyle{empty}

\title{Multidimensional Clustering of Web Analytics Data}

\begin{center}
  {\bf Alexander Kruse$^{1^\star}$}
\end{center}

\vskip 0.3cm

\begin{affiliations}
\begin{enumerate}
\begin{minipage}{0.915\textwidth}
\centering
\item etracker GmbH \\[-2pt]
\end{minipage}
\end{enumerate}
$^\star$Contact author: \href{mailto:kruse@etracker.de}{\nolinkurl{kruse@etracker.de}}\\
\end{affiliations}

\vskip 0.5cm

\begin{minipage}{0.915\textwidth}
\keywords data mining; clustering; web analytics
\packages amap; dplyr; ggplot2
\end{minipage}

\vskip 0.8cm

Web Analytics is the collection of data and their analysis regarding the
behavior of website visitors. This presentation demonstrates a
walkthrough on how to use multidimensional cluster analysis to divide a
heterogeneous group of website visitors into smaller homogeneous
segments. Focus is the appropriate selection of segmentation features,
the determination of number of clusters and the usage of a
multidimensional clustering technique. All preprocessing, analysis and
visualization has been done with the \textbf{R} programming language.

\end{document}
