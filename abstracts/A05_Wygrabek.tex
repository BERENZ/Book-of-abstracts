\providecommand{\main}{..} 
\documentclass[\main/boa.tex]{subfiles}

\begin{document}

\section{Embedding R in business processes}

\begin{center}
  {\bf Andreas Wygrabek$^{1^\star}$}
\end{center}

\vskip 0.3cm

\begin{affiliations}
\begin{enumerate}
\begin{minipage}{0.915\textwidth}
\centering
\item eoda GmbH \\[-2pt]
\end{minipage}
\end{enumerate}
$^\star$Contact author: \href{mailto:andreas.wygrabek@eoda.de}{\nolinkurl{andreas.wygrabek@eoda.de}}\\
\end{affiliations}

\vskip 0.5cm

\begin{minipage}{0.915\textwidth}
\keywords business; process integration
\end{minipage}

\vskip 0.8cm

There is no doubt about it: a variety of implemented statistical methods
that are easy to access can make \textbf{R} a valuable tool for plenty
application scenarios in business e.g.~planning sales campaigns,
monitoring machines, acceptance sampling on incoming goods or in-process
control. All these scenarios are ruled by standardized business
processes and tied with substantial financial risks. Beside the
analytical capabilities of \textbf{R}, there are two main requirements
on \textbf{R} that are important for companies: on the one hand
\textbf{R} needs to be smoothly integrated into existing business
processes. On the other hand, \textbf{R} needs to be a reliable part of
the process chain: secure, easy to maintain, automatable and integrable
into a right management system. Even though \textbf{R} reaches an
incredible popularity many companies struggle to make \textbf{R} more
than a proof of concept tool. The maturity level of \textbf{R} is
reached when it has become crutial a part of business process chains.
The talk will delight the main questions and critical issues of making
\textbf{R} an essential link in a business process chain. Furthermore
the talk will present some state of the art strategies to face the
depicted issues regarding security, reliability and automatization in
operating systems.

\end{document}
