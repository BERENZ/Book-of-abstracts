\providecommand{\main}{..} 
\documentclass[\main/boa.tex]{subfiles}

\begin{document}

\section{Design of Experiments in R}

\begin{center}
  {\bf \index[a]{Grömping Ulrike} Ulrike Grömping$^{1^\star}$}
\end{center}

\vskip 0.3cm

\begin{affiliations}
\begin{enumerate}
\begin{minipage}{0.915\textwidth}
\centering
\item Beuth University of Applied Sciences, Berlin \\[-2pt]
\end{minipage}
\end{enumerate}
$^\star$Contact author: \href{mailto:groemping@bht-berlin.de}{\nolinkurl{groemping@bht-berlin.de}}\\
\end{affiliations}

\vskip 0.5cm

\begin{minipage}{0.915\textwidth}
\keywords design of experiments
\packages \index[p]{agricolae} agricolae; \index[p]{AlgDesign} AlgDesign; \index[p]{conf.design} conf.design; \index[p]{DiceDesign} DiceDesign; \index[p]{DiceKriging} DiceKriging; \index[p]{DoE.base} DoE.base; \index[p]{DoE.wrapper} DoE.wrapper; \index[p]{ICAOD} ICAOD; \index[p]{lhs} lhs; \index[p]{OptimalDesign} OptimalDesign; \index[p]{planor} planor; \index[p]{rsm} rsm; \index[p]{tgp} tgp
\end{minipage}

\vskip 0.8cm

Starting with an example from the Obama electoral campaign, this talk 
initially introduces the main principles of experimental design and 
explains different types of designs suitable for different purposes. 
I then discuss the development of the landscape of packages on
Design of Experiments in \textbf{R}, the current state of which is
documented in the Experimental Design Task View
(\url{http://cran.r-project.org/web/views/ExperimentalDesign.html}).
That landscape is quite diverse, currently consisting of (at least) four
larger areas and various specialized packages. Descendants and relatives
of the pioneering packages \emph{conf.desig}n (2001) and
\emph{AlgDesign} (2004) as well as the growing area of packages for
computer experiments are considered in more detail, and recent additions
based on modern optimization methods are discussed.

\end{document}
