\providecommand{\main}{..} 
\documentclass[\main/boa.tex]{subfiles}

\begin{document}
\pagestyle{empty}

\section{Design of Experiments in R}

\begin{center}
  {\bf Ulrike Grömping$^{1^\star}$}
\end{center}

\vskip 0.3cm

\begin{affiliations}
\begin{enumerate}
\begin{minipage}{0.915\textwidth}
\centering
\item Beuth University of Applied Sciences, Berlin \\[-2pt]
\end{minipage}
\end{enumerate}
$^\star$Contact author: \href{mailto:groemping@bht-berlin.de}{\nolinkurl{groemping@bht-berlin.de}}\\
\end{affiliations}

\vskip 0.5cm

\begin{minipage}{0.915\textwidth}
\keywords design of experiments
\packages agricolae; AlgDesign; conf.design; DiceDesign; DiceKriging; DoE.base;
DoE.wrapper; ICAOD; lhs; OptimalDesign; planor; rsm; tgp
\end{minipage}

\vskip 0.8cm

This talk discusses the development of the landscape of packages on
Design of Experiments in \textbf{R}, the current state of which is
documented in the Experimental Design Task View
(\url{http://cran.r-project.org/web/views/ExperimentalDesign.html}).
That landscape is quite diverse, currently consisting of (at least) four
larger areas and various specialized packages. Descendants and relatives
of the pioneering packages \emph{conf.desig}n (2001) and
\emph{AlgDesign} (2004) as well as the growing area of packages for
computer experiments are considered in more detail, and recent additions
based on modern optimization methods are discussed.

\end{document}
